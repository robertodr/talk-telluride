\documentclass{article}

\usepackage{amsmath}

% For diagrams: https://github.com/avcopan/styfiles/blob/master/goldstone.sty
\usepackage[scale=1.1]{goldstone} % option [scale=<0.9, 1.1, ...>] (default is 0.85)

\usepackage{simplewick}

\begin{document}

\begin{equation*}
  F =
  \diagram{
    \draw (-0.5,0) node[circlex] (f) {} -- (0,0) node[ddot=white] (f1) {};
    \draw[->-] (f1) to node[right] {} ++(0,+0.5);
    \draw[-<-] (f1) to node[right] {} ++(0,-0.5);
  }
  +
  \diagram{
    \draw (-0.5,0) node[circlex] (f) {} -- (0,0) node[ddot=white] (f1) {};
    \draw[-<-] (f1) to node[right] {} ++(0,+0.5);
    \draw[->-] (f1) to node[right] {} ++(0,-0.5);
  }
  +
  \diagram{
    \draw (-0.5,0) node[circlex] (f) {} -- (0,0) node[ddot=white] (f1) {};
    \draw[->-] (f1) to node[left] {} ++(-0.25,-0.5);
    \draw[-<-] (f1) to node[right] {} ++(0.25,-0.5);
  }
  +
  \diagram{
    \draw (-0.5,0) node[circlex] (f) {} -- (0,0) node[ddot=white] (f1) {};
    \draw[-<-] (f1) to node[left] {} ++(-0.25,0.5);
    \draw[->-] (f1) to node[right] {} ++(0.25,0.5);
  }
\end{equation*}

\begin{equation*}
  \begin{aligned}
    \Phi
    &=
    \diagram{
      \interaction{2}{g}{(0,0)}{ddot=white}{flexdotted};
      \draw[->-] (g1) to node[left] {} ++(0,+0.5);
      \draw[-<-] (g1) to node[left] {}++(0,-0.5);
      \draw[->-] (g2) to node[left] {}++(0,+0.5);
      \draw[-<-] (g2) to node[left] {}++(0,-0.5);
    }
    +
    \diagram{
      \interaction{2}{g}{(0,0)}{ddot=white}{flexdotted};
      \draw[-<-] (g1) to node[left] {} ++(0,+0.5);
      \draw[->-] (g1) to node[left] {}++(0,-0.5);
      \draw[-<-] (g2) to node[left] {}++(0,+0.5);
      \draw[->-] (g2) to node[left] {}++(0,-0.5);
    }
    +
    \diagram{
      \interaction{2}{g}{(0,0)}{ddot=white}{flexdotted};
      \draw[->-] (g1) to node[left] {} ++(0,+0.5);
      \draw[-<-] (g1) to node[left] {}++(0,-0.5);
      \draw[-<-] (g2) to node[left] {}++(0,+0.5);
      \draw[->-] (g2) to node[left] {}++(0,-0.5);
    } \\
    &+
    \diagram{
      \interaction{2}{g}{(0,0)}{ddot=white}{flexdotted};
      \draw[->-] (g1) to node[left] {} ++(0,+0.5);
      \draw[-<-] (g1) to node[left] {}++(0,-0.5);
      \draw[-<-] (g2) to node[left] {}++(-0.25,+0.5);
      \draw[->-] (g2) to node[right] {}++(0.25,+0.5);
    }
    +
    \diagram{
      \interaction{2}{g}{(0,0)}{ddot=white}{flexdotted};
      \draw[-<-] (g1) to node[left] {} ++(0,+0.5);
      \draw[->-] (g1) to node[left] {}++(0,-0.5);
      \draw[-<-] (g2) to node[left] {}++(-0.25,+0.5);
      \draw[->-] (g2) to node[right] {}++(0.25,+0.5);
    }
    +
    \diagram{
      \interaction{2}{g}{(0,0)}{ddot=white}{flexdotted};
      \draw[->-] (g1) to node[left] {} ++(0,+0.5);
      \draw[-<-] (g1) to node[left] {}++(0,-0.5);
      \draw[->-] (g2) to node[left] {}++(-0.25,-0.5);
      \draw[-<-] (g2) to node[right] {}++(0.25,-0.5);
    } \\
    &+
    \diagram{
      \interaction{2}{g}{(0,0)}{ddot=white}{flexdotted};
      \draw[-<-] (g1) to node[left] {} ++(0,+0.5);
      \draw[->-] (g1) to node[left] {}++(0,-0.5);
      \draw[->-] (g2) to node[left] {}++(-0.25,-0.5);
      \draw[-<-] (g2) to node[right] {}++(0.25,-0.5);
    }
    +
    \diagram{
      \interaction{2}{g}{(0,0)}{ddot=white}{flexdotted};
      \draw[->-] (g1) to node[left] {} ++(-0.25,-0.5);
      \draw[-<-] (g1) to node[right] {}++(0.25,-0.5);
      \draw[->-] (g2) to node[left] {}++(-0.25,-0.5);
      \draw[-<-] (g2) to node[right] {}++(0.25,-0.5);
    }
    +
    \diagram{
      \interaction{2}{g}{(0,0)}{ddot=white}{flexdotted};
      \draw[-<-] (g1) to node[left] {} ++(-0.25,0.5);
      \draw[->-] (g1) to node[right] {}++(0.25,0.5);
      \draw[-<-] (g2) to node[left] {}++(-0.25,0.5);
      \draw[->-] (g2) to node[right] {}++(0.25,0.5);
    }
  \end{aligned}
\end{equation*}

\begin{equation*}
  T_{1} =
  \diagram{
    \interaction{1}{ta}{(0, -0.5)}{ddot}{overhang};
    \draw[->-] (ta1) to node[left] {} ++(-0.25, 1.0);
    \draw[-<-] (ta1) to node[right] {} ++(0.25, 1.0);
  }
\end{equation*}

\begin{equation*}
  T_{2} =
  \diagram{
    \interaction{2}{ta}{(0, -0.5)}{ddot}{overhang};
    \draw[->-] (ta1) to node[left] {} ++(-0.25, 1.0);
    \draw[-<-] (ta1) to node[right] {} ++(0.25, 1.0);
    \draw[->-] (ta2) to node[left] {} ++(-0.25, 1.0);
    \draw[-<-] (ta2) to node[right] {} ++(0.25, 1.0);
  }
\end{equation*}

\begin{equation*}
  T_{3} =
  \diagram{
    \interaction{3}{ta}{(0, -0.5)}{ddot}{overhang};
    \draw[->-] (ta1) to node[left] {} ++(-0.25, 1.0);
    \draw[-<-] (ta1) to node[right] {} ++(0.25, 1.0);
    \draw[->-] (ta2) to node[left] {} ++(-0.25, 1.0);
    \draw[-<-] (ta2) to node[right] {} ++(0.25, 1.0);
    \draw[->-] (ta3) to node[left] {} ++(-0.25, 1.0);
    \draw[-<-] (ta3) to node[right] {} ++(0.25, 1.0);
  }
\end{equation*}

\begin{equation*}
  T_{1} =
  \diagram{
    \interaction{1}{ta}{(0, -0.5)}{ddot}{overhang};
    \draw[->-, red!70] (ta1) to node[left] {} ++(-0.25, 1.0);
    \draw[-<-, blue!70] (ta1) to node[right] {} ++(0.25, 1.0);
  }
  = t_{\textcolor{blue}{i}}^{\textcolor{red}{a}}\tau_{\textcolor{blue}{i}}^{\textcolor{red}{a}}
\end{equation*}

\begin{equation*}
  T_{2} =
  \diagram{
    \interaction{2}{ta}{(0, -0.5)}{ddot}{overhang};
    \draw[->-, red!70] (ta1) to node[left] {} ++(-0.25, 1.0);
    \draw[-<-, blue!70] (ta1) to node[right] {} ++(0.25, 1.0);
    \draw[->-, red!70] (ta2) to node[left] {} ++(-0.25, 1.0);
    \draw[-<-, blue!70] (ta2) to node[right] {} ++(0.25, 1.0);
  }
  = \frac{1}{(2!)^2} t_{\textcolor{blue}{ij}}^{\textcolor{red}{ab}}\tau_{\textcolor{blue}{ij}}^{\textcolor{red}{ab}}
\end{equation*}

\begin{equation*}
  T_{3} =
  \diagram{
    \interaction{3}{ta}{(0, -0.5)}{ddot}{overhang};
    \draw[->-, red!70] (ta1) to node[left] {} ++(-0.25, 1.0);
    \draw[-<-, blue!70] (ta1) to node[right] {} ++(0.25, 1.0);
    \draw[->-, red!70] (ta2) to node[left] {} ++(-0.25, 1.0);
    \draw[-<-, blue!70] (ta2) to node[right] {} ++(0.25, 1.0);
    \draw[->-, red!70] (ta3) to node[left] {} ++(-0.25, 1.0);
    \draw[-<-, blue!70] (ta3) to node[right] {} ++(0.25, 1.0);
  }
  = \frac{1}{(3!)^2} t_{\textcolor{blue}{ijk}}^{\textcolor{red}{abc}}\tau_{\textcolor{blue}{ijk}}^{\textcolor{red}{abc}}
\end{equation*}

\begin{equation*}
  \contraction{}{F}{}{T_{1}}
  FT_{1}
  =
  \diagram{
    \interaction{1}{ta}{(-0.25, -0.5)}{ddot}{overhang};
    \draw[-<-, semithick] (ta1) to node[left] {} ++(-0.5, 1.0);
    \draw (0.5, 0) node[circlex] (f) {} -- (0, 0) node[ddot=white] (f1) {};
    \draw[->-, semithick] (ta1) to (f1) {};
    \draw[->-, semithick] (f1) to node[left] {} ++(-0.25, 0.5);
  }
  +
  \diagram{
    \interaction{1}{ta}{(0.25, -0.5)}{ddot}{overhang};
    \draw[->-, semithick] (ta1) to node[right] {} ++(+0.5, 1.0);
    \draw (-0.5, 0) node[circlex] (f) {} -- (0, 0) node[ddot=white] (f1) {};
    \draw[-<-, semithick] (ta1) to (f1) {};
    \draw[-<-, semithick] (f1) to node[left] {} ++(0.25, 0.5);
  }
  +
  \diagram{
    \interaction{1}{ta}{(0, -0.5)}{ddot}{overhang};
    \draw (0.5, 0) node[circlex] (f) {} -- (0, 0) node[ddot=white] (f1) {};
    \draw[-<-, bend left] (ta1) to (f1) {};
    \draw[->-, bend right] (ta1) to (f1) {};
  }
  +
  \diagram{
    \interaction{1}{ta}{(-0.25, -0.5)}{ddot}{overhang};
    \draw[-<-, semithick] (ta1) to node[left] {} ++(-0.5, 1.0);
    \draw (0.5, 0) node[circlex] (f) {} -- (0, 0) node[ddot=white] (f1) {};
    \draw[->-, semithick] (ta1) to (f1) {};
    \draw[->-, semithick] (f1) to node[right] {} ++(+0.25, -0.5);
  }
  +
  \diagram{
    \interaction{1}{ta}{(0.25, -0.5)}{ddot}{overhang};
    \draw[->-, semithick] (ta1) to node[right] {} ++(+0.5, 1.0);
    \draw (-0.5, 0) node[circlex] (f) {} -- (0, 0) node[ddot=white] (f1) {};
    \draw[-<-, semithick] (ta1) to (f1) {};
    \draw[-<-, semithick] (f1) to node[left] {} ++(-0.25, -0.5);
  }
\end{equation*}

\begin{equation*}
  \contraction{}{F}{}{T_{1}}
  FT_{1}
  =
  \diagram{
    \interaction{1}{ta}{(-0.25, -0.5)}{ddot}{overhang};
    \draw[-<-, semithick, orange!70] (ta1) to node[left] {} ++(-0.5, 1.0);
    \draw (0.5, 0) node[circlex] (f) {} -- (0, 0) node[ddot=white] (f1) {};
    \draw[->-, semithick, purple!70] (ta1) to (f1) {};
    \draw[->-, semithick, orange!70] (f1) to node[left] {} ++(-0.25, 0.5);
  }
  +
  \diagram{
    \interaction{1}{ta}{(0.25, -0.5)}{ddot}{overhang};
    \draw[->-, semithick, orange!70] (ta1) to node[right] {} ++(+0.5, 1.0);
    \draw (-0.5, 0) node[circlex] (f) {} -- (0, 0) node[ddot=white] (f1) {};
    \draw[-<-, semithick, purple!70] (ta1) to (f1) {};
    \draw[-<-, semithick, orange!70] (f1) to node[left] {} ++(0.25, 0.5);
  }
  +
  \diagram{
    \interaction{1}{ta}{(0, -0.5)}{ddot}{overhang};
    \draw (0.5, 0) node[circlex] (f) {} -- (0, 0) node[ddot=white] (f1) {};
    \draw[-<-, bend left, purple!70] (ta1) to (f1) {};
    \draw[->-, bend right, purple!70] (ta1) to (f1) {};
  }
  +
  \diagram{
    \interaction{1}{ta}{(-0.25, -0.5)}{ddot}{overhang};
    \draw[-<-, semithick, orange!70] (ta1) to node[left] {} ++(-0.5, 1.0);
    \draw (0.5, 0) node[circlex] (f) {} -- (0, 0) node[ddot=white] (f1) {};
    \draw[->-, semithick, purple!70] (ta1) to (f1) {};
    \draw[->-, semithick, orange!70] (f1) to node[right] {} ++(+0.25, -0.5);
  }
  +
  \diagram{
    \interaction{1}{ta}{(0.25, -0.5)}{ddot}{overhang};
    \draw[->-, semithick, orange!70] (ta1) to node[right] {} ++(+0.5, 1.0);
    \draw (-0.5, 0) node[circlex] (f) {} -- (0, 0) node[ddot=white] (f1) {};
    \draw[-<-, semithick, purple!70] (ta1) to (f1) {};
    \draw[-<-, semithick, orange!70] (f1) to node[left] {} ++(-0.25, -0.5);
  }
\end{equation*}

\begin{equation*}
  \begin{aligned}
  \contraction{}{F}{}{T_{1}}
  FT_{1}|D_{\mathbf{0}}\rangle
  &=
  \diagram{
    \interaction{1}{ta}{(-0.25, -0.5)}{ddot}{overhang};
    \draw[-<-, semithick, orange!70] (ta1) to node[left] {} ++(-0.5, 1.0);
    \draw (0.5, 0) node[circlex] (f) {} -- (0, 0) node[ddot=white] (f1) {};
    \draw[->-, semithick, purple!70] (ta1) to (f1) {};
    \draw[->-, semithick, orange!70] (f1) to node[left] {} ++(-0.25, 0.5);
  }
  +
  \diagram{
    \interaction{1}{ta}{(0.25, -0.5)}{ddot}{overhang};
    \draw[->-, semithick, orange!70] (ta1) to node[right] {} ++(+0.5, 1.0);
    \draw (-0.5, 0) node[circlex] (f) {} -- (0, 0) node[ddot=white] (f1) {};
    \draw[-<-, semithick, purple!70] (ta1) to (f1) {};
    \draw[-<-, semithick, orange!70] (f1) to node[left] {} ++(0.25, 0.5);
  }
  +
  \diagram{
    \interaction{1}{ta}{(0, -0.5)}{ddot}{overhang};
    \draw (0.5, 0) node[circlex] (f) {} -- (0, 0) node[ddot=white] (f1) {};
    \draw[-<-, bend left, purple!70] (ta1) to (f1) {};
    \draw[->-, bend right, purple!70] (ta1) to (f1) {};
  } \\
  &=
    f_{\textcolor{orange}{a}}^{\textcolor{purple}{b}}t_{\textcolor{orange}{i}}^{\textcolor{purple}{b}}\tau_{\textcolor{orange}{i}}^{\textcolor{orange}{a}}
  - f_{\textcolor{purple}{j}}^{\textcolor{orange}{i}}t_{\textcolor{purple}{j}}^{\textcolor{orange}{a}}\tau_{\textcolor{orange}{i}}^{\textcolor{orange}{a}}
  + f_{\textcolor{purple}{i}}^{\textcolor{purple}{a}}t_{\textcolor{purple}{i}}^{\textcolor{purple}{a}}
  \end{aligned}
\end{equation*}

\begin{equation*}
\diagram{
  \interaction{3}{ta}{(0, -0.5)}{ddot}{overhang};
  \draw[->-] (ta1) to node[left] {} ++(-0.25, 1.0);
  \draw[-<-] (ta1) to node[right] {} ++(0.25, 1.0);
  \draw[->-] (ta2) to node[left] {} ++(-0.25, 1.0);
  \draw[-<-] (ta2) to node[right] {} ++(0.25, 1.0);
  \draw[->-] (ta3) to node[left] {} ++(-0.25, 1.0);
  \draw[-<-] (ta3) to node[right] {} ++(0.25, 1.0);
}
  \longleftarrow
\diagram{
  \interaction{1}{ta}{(0, -0.5)}{ddot}{overhang};
  \interaction{1}{tb}{(1, -0.5)}{ddot}{overhang};
  \interaction{3}{tc}{(2, -0.5)}{ddot}{overhang};
  \draw[flexdotted] (0.5, 0.0) node[ddot=white] (g1) {} to (2, 0.0) node[ddot=white] (g2) {};
  \draw[semithick, -<-] (ta1) to node[left] {} ++(-0.25, 1.0);
  \draw[semithick, ->-] (ta1) to node[left] {} (g1);
  \draw[semithick, ->-] (tb1) to node[right] {} ++(0.25, 1.0);
  \draw[semithick, -<-] (tb1) to node[right] {} (g1);
  \draw[semithick, bend left, -<-] (tc1) to node[right] {} (g2);
  \draw[semithick, bend right, ->-] (tc1) to node[right] {} (g2);
  \draw[semithick, ->-] (tc2) to node[right] {} ++(0.25, 1.0);
  \draw[semithick, -<-] (tc2) to node[right] {} ++(-0.25, 1.0);
  \draw[semithick, ->-] (tc3) to node[right] {} ++(0.25, 1.0);
  \draw[semithick, -<-] (tc3) to node[right] {} ++(-0.25, 1.0);
}
\end{equation*}

\begin{equation*}
  \diagram{
    \interaction{1}{ta}{(0, -0.5)}{ddot}{overhang};
    \draw[->-] (ta1) to node[left] {} ++(-0.25, 1.0);
    \draw[-<-] (ta1) to node[right] {} ++(0.25, 1.0);
  }
  \longleftarrow
  \diagram{
    \interaction{1}{ta}{(0, -0.5)}{ddot}{overhang};
    \draw[->-] (ta1) to node[left] {} ++(-0.25, 1.0);
    \draw[-<-] (ta1) to node[right] {} ++(0.25, 1.0);
  }
  - \delta\tau
  \diagram{
    \interaction{1}{ta}{(0, -0.5)}{ddot=white}{};
    \draw[wiggly] (0, -0.5) node (g1) {} to (0.75, -0.5) node (g2) {};
    \draw[->-] (ta1) to node[left] {} ++(-0.25, 1.0);
    \draw[-<-] (ta1) to node[right] {} ++(0.25, 1.0);
  }
\end{equation*}

\begin{equation*}
  \diagram{
    \interaction{2}{ta}{(0, -0.5)}{ddot}{overhang};
    \draw[->-] (ta1) to node[left] {} ++(-0.25, 1.0);
    \draw[-<-] (ta1) to node[right] {} ++(0.25, 1.0);
    \draw[->-] (ta2) to node[left] {} ++(-0.25, 1.0);
    \draw[-<-] (ta2) to node[right] {} ++(0.25, 1.0);
  }
  \longleftarrow
  \diagram{
    \interaction{2}{ta}{(0, -0.5)}{ddot}{overhang};
    \draw[->-] (ta1) to node[left] {} ++(-0.25, 1.0);
    \draw[-<-] (ta1) to node[right] {} ++(0.25, 1.0);
    \draw[->-] (ta2) to node[left] {} ++(-0.25, 1.0);
    \draw[-<-] (ta2) to node[right] {} ++(0.25, 1.0);
  }
  - \delta\tau
  \diagram{
    \interaction{2}{ta}{(0, -0.5)}{ddot=white}{wiggly};
    \draw[->-] (ta1) to node[left] {} ++(-0.25, 1.0);
    \draw[-<-] (ta1) to node[right] {} ++(0.25, 1.0);
    \draw[->-] (ta2) to node[left] {} ++(-0.25, 1.0);
    \draw[-<-] (ta2) to node[right] {} ++(0.25, 1.0);
  }
\end{equation*}

\begin{equation*}
  \diagram{
    \interaction{3}{ta}{(0, -0.5)}{ddot}{overhang};
    \draw[->-] (ta1) to node[left] {} ++(-0.25, 1.0);
    \draw[-<-] (ta1) to node[right] {} ++(0.25, 1.0);
    \draw[->-] (ta2) to node[left] {} ++(-0.25, 1.0);
    \draw[-<-] (ta2) to node[right] {} ++(0.25, 1.0);
    \draw[->-] (ta3) to node[left] {} ++(-0.25, 1.0);
    \draw[-<-] (ta3) to node[right] {} ++(0.25, 1.0);
  }
  \longleftarrow
  \diagram{
    \interaction{3}{ta}{(0, -0.5)}{ddot}{overhang};
    \draw[->-] (ta1) to node[left] {} ++(-0.25, 1.0);
    \draw[-<-] (ta1) to node[right] {} ++(0.25, 1.0);
    \draw[->-] (ta2) to node[left] {} ++(-0.25, 1.0);
    \draw[-<-] (ta2) to node[right] {} ++(0.25, 1.0);
    \draw[->-] (ta3) to node[left] {} ++(-0.25, 1.0);
    \draw[-<-] (ta3) to node[right] {} ++(0.25, 1.0);
  }
  - \delta\tau
  \diagram{
    \interaction{3}{ta}{(0, -0.5)}{ddot=white}{wiggly};
    \draw[->-] (ta1) to node[left] {} ++(-0.25, 1.0);
    \draw[-<-] (ta1) to node[right] {} ++(0.25, 1.0);
    \draw[->-] (ta2) to node[left] {} ++(-0.25, 1.0);
    \draw[-<-] (ta2) to node[right] {} ++(0.25, 1.0);
    \draw[->-] (ta3) to node[left] {} ++(-0.25, 1.0);
    \draw[-<-] (ta3) to node[right] {} ++(0.25, 1.0);
  }
\end{equation*}

\end{document}
